\documentclass[12pt,letterpaper]{article}
\usepackage[spanish,es-nodecimaldot]{babel}
\usepackage[utf8]{inputenc}
\usepackage[T1]{fontenc}
\usepackage{float}

%% Sets page size and margins
\usepackage[leter,top=2.5cm,bottom=2.5cm,left=2cm,right=2cm]{geometry}
\usepackage{bussproofs}
\usepackage{amsmath}
\usepackage{amssymb}
\usepackage{amsthm}
\usepackage{mathtools}
\usepackage{listings}
\usepackage{enumitem}[shortlabels]
\usepackage[colorinlistoftodos]{todonotes}
\usepackage[colorlinks=true, allcolors=blue]{hyperref}
\usepackage{url}

\usepackage{lipsum}

\usepackage{tcolorbox}

%Author affil
\usepackage{authblk}

%% Title
\title{
		\vspace{-0.7in} 	
		\usefont{OT1}{bch}{b}{n}
		
		\begin{minipage}{3cm}
        \vspace{-0.5in} 	
    	\begin{center}
    		\includegraphics[height=3.2cm]{unam.png}
    	\end{center}
    \end{minipage}\hfill
    \begin{minipage}{10.7cm}
    	\begin{center}

\normalfont \normalsize \textsc{UNIVERSIDAD NACIONAL AUTÓNOMA DE MÉXICO \\ FACULTAD DE CIENCIAS \\ LENGUAJES DE PROGRAMACIÓN } \\
		\huge Semanal 03

    	\end{center}
     
    \end{minipage}\hfill
    \begin{minipage}{3.2cm}
    \vspace{-0.5in} 
    	\begin{center}
    		\includegraphics[height=3.2cm]{fc_logo.png}
    	\end{center}
    \end{minipage}

\author{Martínez Osorio Benjamín 312063678 \\Salazar Gonzalez Pedro Yamil 306037445}
\date{}

}




\hypersetup{
    colorlinks=true,
    linkcolor=blue,
    filecolor=magenta,      
    urlcolor=cyan
    }


%------Vectores Unitarios-------%
\newcommand{\uveci}{\hat{\textbf{i}}} 
\newcommand{\uvecj}{\hat{\textbf{j}}}
\newcommand{\uveck}{\hat{\textbf{k}}}

\newcommand{\grade}{^{\circ}}
 \setlength {\marginparwidth }{2cm}
 
\begin{document}

\maketitle

\section*{Ejercicio 1}
\textbf{\large $( \hspace{0.5cm}-\hspace{0.5cm} (\hspace{0.5cm} +\hspace{0.5cm} 20\hspace{0.5cm} 3\hspace{0.5cm})\hspace{0.5cm} (\hspace{0.5cm} -\hspace{0.5cm} -18\hspace{0.5cm} (\hspace{0.5cm} +\hspace{0.5cm} 50\hspace{0.5cm} 20\hspace{0.5cm})\hspace{0.5cm} ) \hspace{0.5cm}) $}\newline

\subsection*{(a)Sintaxis Abstracta:}

$Sub(Add(Num(20), Num(3)) Sub(Num(-18), Add(Num(50), Num(20))))$
\subsection*{(b)Evaluación Natural:}

    \begin{prooftree}
        \AxiomC{Num(20) $\Rightarrow$ Num(20)} \AxiomC{Num(3) $\Rightarrow$ Num(3)}
            \BinaryInfC{Add(Num(20), Num(3)) $\Rightarrow$ Num(23)}
                \AxiomC{Num(-18) $\Rightarrow$ Num(-18)} \AxiomC{Num(50) $\Rightarrow$ Num(50)}
                                                        \AxiomC{Num(20) $\Rightarrow$ Num(20)}
                                                \BinaryInfC{Add(Num(50), Num(20)) $\Rightarrow$ Num(70)}
                                    \BinaryInfC{Sub(Num(-18), Add(Num(50), Num(20)) $\Rightarrow$ Num(-88))}
                        \BinaryInfC{Sub(Add(Num(20), Num(3)), Sub(Num(-18), Add(Num(50), Num(20)))) $\Rightarrow$ Num(111)}
    \end{prooftree}

        \begin{figure}
            \centering
            \includegraphics[width=0.85\textwidth]{evnatural_1.jpg}
            \caption{Evaluación Natural}
            \label{fig:enter-label}
        \end{figure}
    \newpage
    \subsection*{(c) Evaluación Estructural}


\begin{align*}
&\text{Sub}\left(\text{Add}\left(\text{Num}(20), \text{Num}(3)\right), \text{Sub}\left(\text{Num}(-18), \text{Add}\left(\text{Num}(50), \text{Num}(20)\right)\right)\right) \\
&\rightarrow \text{Sub}\left(\text{Num}(20 + 3), \text{Sub}\left(\text{Num}(-18), \text{Num}(50 + 20)\right)\right) \\
&\rightarrow \text{Sub}\left(\text{Num}(23), \text{Sub}\left(\text{Num}(-18), \text{Num}(70)\right)\right) \\
&\rightarrow \text{Sub}\left(\text{Num}(23), \text{Sub}\left(\text{Num}(-18 - 70)\right)\right) \\
&\rightarrow \text{Sub}\left(\text{Num}(23), \text{Num}(-88)\right) \\
&\rightarrow \text{Num}(23 - -88) = \text{Num}(111)
\end{align*}
Resultado: \(\mathbf{111}\)





\section*{Ejercicio 2:} 
\texttt{\large(not (+ 1 (- 3 (+ -8 1))))}

\subsection*{(a) Sintaxis Abstracta}
\[
\text{Not}(\text{Add}(1, \text{Sub}(3, \text{Add}(-8, 1))))
\]

\subsection*{(b) Evaluación usando Semántica Natural}

\begin{align*}
1. &\quad (+ -8\ 1) \rightarrow -7 \\
2. &\quad (- 3\ -7) \rightarrow 10 \\
3. &\quad (+ 1\ 10) \rightarrow 11 \\
4. &\quad \text{not } 11 \rightarrow \text{false}
\end{align*}

Resultado: \(\mathbf{false}\)

\subsection*{(c) Evaluación usando Semántica Estructural}

\begin{align*}
1. &\quad (+ -8\ 1) \rightarrow -8 + 1 = -7 \\
2. &\quad (- 3\ (+ -8\ 1)) \rightarrow 3 - (-7) = 10 \\
3. &\quad (+ 1\ (- 3\ (+ -8\ 1))) \rightarrow 1 + 10 = 11 \\
4. &\quad \text{not } (+ 1\ (- 3\ (+ -8\ 1))) \rightarrow \text{not } 11 = \text{false}
\end{align*}

Resultado: \(\mathbf{false}\)

\vspace{0.5cm}

\section*{Ejercicio 3:} 
\texttt{\large(not (not (+ 3 5)))}

\subsection*{(a) Sintaxis Abstracta}
\[
\text{Not}(\text{Not}(\text{Add}(3, 5)))
\]

\subsection*{(b) Evaluación usando Semántica Natural}

\begin{align*}
1. &\quad (+ 3\ 5) \rightarrow 8 \\
2. &\quad \text{not } 8 \rightarrow \text{false} \\
3. &\quad \text{not } \text{false} \rightarrow \text{true}
\end{align*}

Resultado: \(\mathbf{true}\)

\subsection*{(c) Evaluación usando Semántica Estructural}

\begin{align*}
1. &\quad (+ 3\ 5) \rightarrow 3 + 5 = 8 \\
2. &\quad \text{not } 8 \rightarrow \text{not } 8 = \text{false} \\
3. &\quad \text{not } (\text{not } 8) \rightarrow \text{not } \text{false} = \text{true}
\end{align*}

Resultado: \(\mathbf{true}\)

\section*{Extensión de la Batería de Operaciones en MiniLisp}

\textbf{Nuevos constructores:}
\begin{itemize}
    \item \texttt{*} para la multiplicación binaria de expresiones aritméticas.
    \item \texttt{/} para la división binaria de expresiones aritméticas (con verificación de división por cero).
    \item \texttt{add1} para incrementar en uno el valor de una expresión.
    \item \texttt{sub1} para decrementar en uno el valor de una expresión.
    \item \texttt{sqrt} para calcular la raíz cuadrada de una expresión (con verificación de valores negativos).
\end{itemize}

\subsection*{(a) Gramática en EBNF}

La gramática libre de contexto en notación EBNF, incluyendo los nuevos constructores:

\begin{verbatim}
<expr> ::= <num>
        | '(' <op> <expr> <expr> ')'
        | '(' <unary-op> <expr> ')'

<num> ::= [0-9]+

<op> ::= '+' | '-' | '*' | '/' 

<unary-op> ::= 'not' | 'add1' | 'sub1' | 'sqrt'
\end{verbatim}

Se agrega unary-op dado que el not, add1, sub1 y sqrt pueden funcionar para la expresión con sólo un número

\subsection*{(b) Reglas de Sintaxis Abstracta}

Las reglas de sintaxis abstracta se modifican para incluir los nuevos constructores:

\[
\begin{aligned}
\text{Expr} &::= \text{Num}(n) \\
            &\quad | \text{Add}(\text{Expr}, \text{Expr}) \\
            &\quad | \text{Sub}(\text{Expr}, \text{Expr}) \\
            &\quad | \text{Mul}(\text{Expr}, \text{Expr}) \\
            &\quad | \text{Div}(\text{Expr}, \text{Expr}) \\
            &\quad | \text{Add1}(\text{Expr}) \\
            &\quad | \text{Sub1}(\text{Expr}) \\
            &\quad | \text{Sqrt}(\text{Expr}) \\
            &\quad | \text{Not}(\text{Expr})
\end{aligned}
\]

\subsection*{(c) Extensión de las Reglas de Semántica}

\subsubsection*{Semántica Natural}

Las reglas de semántica natural para los nuevos operadores:

\[
\frac{e_1 \rightarrow n_1 \quad e_2 \rightarrow n_2}{(Mul\ e_1\ e_2) \rightarrow n_1 \times n_2}
\]

\[
\frac{e_1 \rightarrow n_1 \quad e_2 \rightarrow n_2 \quad n_2 \neq 0}{(Div\ e_1\ e_2) \rightarrow n_1 \div n_2} \quad \text{Error si } n_2 = 0
\]

\[
\frac{e \rightarrow n}{(\text{add1}\ e) \rightarrow n + 1}
\]

\[
\frac{e \rightarrow n}{(\text{sub1}\ e) \rightarrow n - 1}
\]

\[
\frac{e \rightarrow n \quad n \geq 0}{(\text{sqrt}\ e) \rightarrow \sqrt{n}} \quad \text{Error si } n < 0
\]

\subsubsection*{Semántica Estructural}

Las reglas de semántica estructural para los nuevos operadores:

\[
(Mul\ e_1\ e_2) \rightarrow \begin{cases} 
n_1 \times n_2 & \text{si } e_1 \rightarrow n_1 \text{ y } e_2 \rightarrow n_2 \\
\text{error} & \text{si alguna evaluación falla}
\end{cases}
\]

\[
(Div\ e_1\ e_2) \rightarrow \begin{cases} 
n_1 \div n_2 & \text{si } e_1 \rightarrow n_1 \text{ y } e_2 \rightarrow n_2 \text{ y } n_2 \neq 0 \\
\text{error: división entre cero} & \text{si } n_2 = 0 \\
\text{error} & \text{si alguna evaluación falla}
\end{cases}
\]

\[
(\text{add1}\ e) \rightarrow \begin{cases}
n + 1 & \text{si } e \rightarrow n \\
\text{error} & \text{si la evaluación falla}
\end{cases}
\]

\[
(\text{sub1}\ e) \rightarrow \begin{cases}
n - 1 & \text{si } e \rightarrow n \\
\text{error} & \text{si la evaluación falla}
\end{cases}
\]

\[
(\text{sqrt}\ e) \rightarrow \begin{cases}
\sqrt{n} & \text{si } e \rightarrow n \text{ y } n \geq 0 \\
\text{error: raíz negativa} & \text{si } n < 0 \\
\text{error} & \text{si la evaluación falla}
\end{cases}
\]


\end{document}